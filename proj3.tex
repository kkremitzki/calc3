\documentclass{article}
\usepackage{amsmath, graphicx, sidecap, wrapfig, slashbox, booktabs, caption, subcaption}
\DeclareGraphicsExtensions{.jpg,.png}
\begin{document}
\title{Project 3: Riemann Sums of a Double Integral}
\author{Kurt Kremitzki\\
        Calculus III\\
        Spring 2014}
\maketitle
% Comments?

%I want to talk about double integrals, expressed as $\iint\limits_R f(x,y)\mathrm{d}A= \int_c^d\int_a^b f(x,y) \mathrm{d}x \mathrm{d}y$.
My objective in this project is to explore double integrals as a limit of Riemann prisms.

\begin{equation}
    \iint\limits_R f(x,y) \; \mathrm{dA} = \int_c^d \int_a^b f(x,y) \;\mathrm{d}x\;\mathrm{d}y = \lim_{m,n \to \infty} \sum_{i=1}^{m} \sum_{j=1}^{n} f(x_{ij}^{ *}, y_{ij}^{ *}) \Delta x \Delta y
\end{equation}

In particular, I will be examining a function over a region defined as:

\begin{equation}
    f(x,y) = 4 - x^2 + y \;,
    \quad\quad\quad\quad R = [-2,2] \times [0,2]
\end{equation}
\begin{figure}
    \centering
    \includegraphics[scale=0.7]{region.png}
    \caption{The region R of integration.}
\end{figure}


% Show xy-region of integration with intervals in 2d
% Show rectangular prisms approximating surface in 3d
\section{Approximation with Rectangular Prisms}

%\begin{SCfigure}
    %\centering
    %\includegraphics[scale=0.1]{knot_g10}
    %\caption{Awesome Image}
%\end{SCfigure}

   I would like to use 8 rectangular prisms to approximate the volume of the solid region under
$f(x,y)$ over the rectangular region $R$. From the definition of a double integral, I would like to say that
$m = 4$ and $n = 2$, or that I will have 4 partitions of my $x$-interval and 2 partitions of my $y$-interval to
constitute the 8 prisms. The limits of integration over $R$ are $-2 \leq x \leq 2$ and $0 \leq y \leq 2$, so
\\
\begin{equation}
    \Delta A = \Delta x \Delta y = (\frac{b - a}{m})(\frac{d - c}{n}) = (\frac{2 - (-2)}{4})(\frac{2 - 0}{2}) = 1
\end{equation}

Because $\Delta A = 1$, my approximation will be 

\begin{equation}
    V \approx  \sum_{i=1}^{4} \sum_{j=1}^{2} f(x_{ij}^{ *}, y_{ij}^{ *})
\end{equation}

or, in other words, the volume will be approximately equal to the sum of the function evaluated at the sample points in my partitions, in this case.
Because it is generally more accurate, and there is no need for the precision of, say, Simpson's Rule, I will
opt for the Midpoint Rule in choosing my sample points. The sample points and the values for $f(x,y)$ are summarized in Table \ref{tab:table1}.


Summing these values, $V \approx 30$. 


\begin{SCfigure}
    \centering
    \includegraphics[scale=0.6]{surface}
    \caption{The surface $z = f(x,y)$ with 8 Riemann prisms approximating a volume between $f$ and $R$.}
\end{SCfigure}
\begin{table}
\centering
\begin{tabular}{|l||*{4}{c|}}\hline
    \backslashbox{y}{x}
    &\makebox[3em]{-1.5}&\makebox[3em]{-0.5}&\makebox[3em]{0.5}
    &\makebox[3em]{1.5}\\\hline
    0.5 & 2.25 & 4.25 & 4.25 & 2.25 \\\hline
    1.5 & 3.25 & 5.25 & 5.25 & 3.25 \\\hline
\end{tabular}
\caption{$f(x,y)$ at sample points.}
\label{tab:table1}
\end{table}

% Show f(x,y) = 4 - x² + y as a surface in 3d
\section{Volume as a Double Integral}
    Because our particular function $f(x,y) \geq 0$ over the region of integration \\
    $R = [-2, 2] \times [0, 2]$, we can
    interpret the double integral over this region as the volume under the surface $z = f(x, y)$. Therefore, 
\begin{gather}
    \notag{V = \int_{-2}^2 \int_0^2 (4 - x^2 + y) \, \mathrm{d}y \, \mathrm{d}x = \int_{-2}^2 \Big[ 4y - x^2y + \frac{y^2}{2} \Big]_{y = 0}^{y = 2} \: \mathrm{d}x} \\
    \notag{= \int_{-2}^2 10 - 2x^2\; \mathrm{d}x = 10x - \frac{2}{3}x^3 \Big|_{-2}^{2} = \Bigg[\Big( 20 - \frac{2}{3}(8)\Big) - \Big(-20 - \frac{2}{3}(-8)\Big)\Bigg]} \\
    \notag{= 40 - \frac{32}{3} = 29\frac{1}{3}}
\end{gather}
Depending on how accurately we needed to know the volume, the value obtained from sampling midpoints of $f$ may be as close as we'd like, compared to the exact value
found in the double integral.

\section{Varying Approximations of Volume}

If I were to increase the number of prisms in the $y$-direction, my approximation to the volume would remain 30. 
The reason for this has to do with these equations:
\begin{gather}
    \notag{\frac{\partial f}{\partial x} = -2x} \quad\quad\quad\quad \notag{\frac{\partial f}{\partial y} = 1}
\end{gather}

Because the rate of change of $f$ along the $y$-axis is 1, a constant, this value does not depend on a particular value
of $y$, and so we are not necessarily missing information about the behavior of $f$ due to the dearth of sample points in $y$ in the
same way we are in $x$ given that $\frac{\partial f}{\partial x} = -2x$. 

\begin{figure}
    \centering
    \begin{subfigure}{.4\textwidth}
        \centering
        \includegraphics[scale=0.5]{30-x}
    \end{subfigure}
    \begin{subfigure}{.4\textwidth}
        \centering
        \includegraphics[scale=0.5]{30-y}
    \end{subfigure}
\caption{Approximating $V$ when (a) $m=30$ and (b) $n = 30$.}
\end{figure}
\pagebreak
This is readily apparent when I alternately allow $m = 30$ and $n = 30$
in Figure 3. Note how the abundance of prisms in the $y$-direction does not appear to fill in the gaps beneath the
surface; in comparison with Figure 2, there is just as much empty space below and above $f$. 

This is not the case when there are many prisms in the $x$-direction, as we can see that the volume of the prisms begins to approach the true volume under $f$.
In fact, when $m = 30\, , \, n = 2$, $V \approx 29.3452$, which is quite close to the actual value $V = 29\frac{1}{3}$.\\
\end{document}
